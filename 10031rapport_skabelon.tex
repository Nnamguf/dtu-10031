% Figurerne i denne skabelon er fra www.xkcd.com
%
% Brug denne, hvis der printes på begge sider!
\documentclass[a4paper,twoside]{article}
% Brug denne, hvis der kun printes på den ene side!
%\documentclass[a4paper]{article}
\usepackage[utf8]{inputenc}
\usepackage{amsmath}
\usepackage{amsfonts}
\usepackage{amssymb}
\usepackage[danish]{babel} 
\usepackage{graphicx}
\usepackage{subfigure}
% geometry-pakken bruges til f.eks. at ændre margnerne.
% Gør ikke margnerne smallere end dette - Jakob bliver ikke glad!
\usepackage[rmargin=3.2cm,tmargin=3.5cm]{geometry}
% Disse to linjer sikrer, at der bliver mellemrum i stedet for indrykning ved afsnit
\setlength{\parindent}{0pt}
\setlength{\parskip}{1ex plus 0.5ex minus 0.2ex}

% Her begynder selve dokumentet
\begin{document}

% titlepage-miljøet bruges til at sætte en forside op og gør blandt andet at der ikke kommer sidetal
% på og at der automatisk bliver skiftet til en ny side når den er færdig
% At opsætte en forside manuelt kan godt blive lidt kompliceret at se på og man vil derfor ofte
% rykke kode som dette ud i en seperat fil f.eks. forside.tex og inkludere den her med \input{forside}
\begin{titlepage}
\centering
% For forklaring af vspace og stretch, se side 115 i ".. not so short .."
% www.ctan.org/tex-archive/info/lshort/english/lshort.pdf
\vspace*{\stretch{1}}
% rule laver en linje på tværs af siden
\rule{\textwidth}{1mm}\\
\vspace{1cm}
% Her vælger en kæmpe skrifttype og fed skrift
\Huge\bfseries Titlen på rapporten skrives her\\
\vspace{0.7cm}
\rule{\textwidth}{1mm}\\
\vspace{3cm}
\large af\\
Oskar Fugmann (s224073)\\
Mathias Grand (s224070)\\


\vspace{0.7cm}


Gruppenummer: 3
\vspace*{\stretch{2}}
% Nulstiller skriftstørrelse og type (f.eks. fed)
\normalsize
\begin{flushleft}
Rapport i kurset Introduktion til Fysik og nanoteknologi, 10031\\
Danmarks Tekniske Universitet\\
2. Oktober 2022
\end{flushleft}
\end{titlepage}

% Indsætter indholdsfortegelse
\tableofcontents
% Ingen sidetal på siden med indholdsfortegnelsen
\thispagestyle{empty} 
\newpage % Tvunget sideskift for at få indholdsfortegnelsen til at stå på en side for sig selv

\section{Røntgenstråling}
% Tvinger sidetællingen til at starte med 1 på denne side
\setcounter{page}{1} 
Røntgentstårling er elektromagnetisk stråling med en bølgelængde\
$\lambda \backsim 1\cdot 10^{-10}$$m$\\
Røntgen ståling bliver skabt bed 

\begin{figure}
\begin{centering}
\includegraphics[width=0.9\columnwidth]{nerd_sniping.png}
\par\end{centering}
\caption{\label{cap:nerd_sniping}Et eksempel på en figur. Husk at figuren skal indeholde en selvforklarende figurtekst.}
\end{figure}

\section{Endnu et afsnit}

Dette er et eksempel på, hvordan man laver en reference til en artikel \cite{collins2000}. Skulle man finde på at skrive en formel inde i et afsnit, gøres det sådan: $E=mc^2$. Eksempler på formler som står på deres egen linje kender I fra introduktionen til \LaTeX:

% Formel med ligningsnummer
\begin{equation}
\underline{\underline{\frac{d\omega}{dt}}}=\frac{\omega_1 - \omega_0}{t_1 - t_0}=\frac{0-\omega_0}{t_1}=\underline{\underline{-\frac{\omega_0}{t_1}}}
\end{equation}

% Formel uden ligningsnummer
\[
\int^{\omega(t)}_{\omega_0}d\omega=-\int^t_0\frac{\omega_0}{t_1}dt\Rightarrow
\]

Skulle man få lyst brug for at referere til en ligning, skal man give ligningen et navn som det er gjort i ligning~\eqref{eksempelligning}:
\begin{equation}
a^2+b^2=c^2 \label{eksempelligning}
\end{equation}

Bemrk, at det nogle gange er nødvendigt at køre \texttt{pdflatex} to gange, før alle ligningsnumre virker. Det skyldes at \LaTeX\ læser filen fra toppen mod bunden, så hvis ligningen står efter referencen, vil referencen først blive rigtig anden gang \texttt{pdflatex} køres.

\section{Flere figurer ved siden af hinanden}

En ting som man af og til vil få brug for, specielt hvis man skriver noget med en sideantalsbegrænsning, er at sætte flere figurer ved siden af hinanden. Generelt er det noget som man bør prøve ikke at gøre for ofte, men der kan være tilfælde hvor det er nødvendigt.

\begin{figure}
\begin{centering}
\includegraphics[height=4cm]{certainty.png}
\hspace{0.5cm}
\includegraphics[height=4cm]{grapefruit_sucks.png}
\par\end{centering}
\caption{\label{cap:2ien}Her er der blot brugt to
  \texttt{includegraphics}-kommandoer i et figurmiljø. Fordelen er at
  det er simpelt, ulempen er at man kun kan referere til begge figurer
  samtidig og at man kun kan lave en samlet figurtekst.  Referer til
  underfigurer med ord som "`til venstre"' og "`til højre"', eller
  indsæt bogstaver a og b i figurerne, og brug bogstaver til at
  referere til delfigurer.}
\end{figure}

Der er to forskellige måder at gøre det på, som forklaret i
figurteksterne. Figur \ref{cap:2ien} viser den simple måde, men to
\texttt{includegraphics}-kommandoer i et figurmiljø. Figur
\ref{fig:subfigure} viser subfigure metoden, som giver mig flere
muligheder hvad angår figurtekster og referencer. Her kan jeg
f.eks. referere direkte til en af underfigurer \ref{fig:subfigcert} og
\ref{fig:subfiggrape}. Begge metoder kan sagtens udvides til mere en 2
figurer, men vær forsigtige og husk at holde øje med hvor små tingene
i jeres figurer bliver.

\begin{figure}
\centering
\subfigure[Figurtekst til den ene figur.]{
\includegraphics[height=4cm]{certainty.png}
\label{fig:subfigcert}
}
\subfigure[Figurtekst til den anden figur.]{
\includegraphics[height=4cm]{grapefruit_sucks.png}
\label{fig:subfiggrape}
}
\label{fig:subfigure}
\caption{Samlet figur tekst for begge figurer. Ved at bruge
  \texttt{subfigure} kan jeg have en figurtekst til hver af figurerne
  og en samlet og jeg kan også referere til blot den ene eller den
  anden af dem eller samlet. Jeg kan endda referere til underfigurerne
  med \texttt{subref}-kommandoen som figur \subref{fig:subfigcert} og
  \subref{fig:subfiggrape}.  Det er en ulempe ved denne metode at man
  normalt ikke ønsker både separate og en fælles figurtekst, og at det
derfor nemt giver et rodet indtryk.}
\end{figure}

\section{Opgaver}
\subsection{\bold{Opgave 1}}
\subsubsection*{Hvad er energien af en elektron, efter at den krydset et anode/katodegab med et spændingsfald på 20 kV? -Antag, at elektronen starter fra hvile, og angiv resultatet i
både elektronvolt og joule?}
Definationen af en elektronvolt er den energi en elektron opnår, når den passere et spændingsfald på $1V$. Derfor må elektronen opnå en energi på $20keV$. For at omdanne det til joule skal vi multiplicerer med $1,602\cdot10^{-19} \frac{J}{eV}$, som er omskrivningsfaktoren
\begin{equation*}
20\cdot10^3 eV \cdot1,602\cdot10^{-19} \frac{J}{eV}=3.2\cdot10^{-15} J
\end {equation*}

\subsection{\bold{Opgave 2}}
\subsubsection*{Hvad er hastigheden af en elektron med den ovenfor beregnede energi?}
Den ovenfor beregnede energi er elektronens kinetiske energi, og vi kan derfor finde dens fart som $v=\sqrt{\frac{2E}{m}}$.
Det vides at elektronens masse er $9,11\cdot10^{-31} kg$

\begin {equation*}
v=\sqrt{\frac{2\cdot3.2\cdot10^{-15}J}{9.11\cdot10^{-31}kg}}=8.4\cdot10^7 \frac{m}{s}
\end{equation*}
\subsection{\bold{Opgave 4}}
\subsection{\bold{Opgave 5}}
Vi løser opgaven med denne formel $\frac{I(z)}{I_0}=e^{-\mu z}$. vi finder $\mu\over\rho$ i Nist til $2,032\cdot 10^{-1} \frac{g}{cm^2}$ og densiteten af Polymethyl Methacrylate til at være $1,18\frac{cm^3}{g}$\\
$\mu$ findes
\begin{equation*}
    \mu=2,032\cdot10^{-1}\frac{g}{cm^2}\cdot1,18\frac{cm^3}{g}=0,24cm^{-1}
\end{equation*}
$I(z)\over Z_0$ findes
\begin{equation*}
    \frac{I(z)}{I_0}=e^{-0,24cm^{-1}\cdot 1cm}=0,79
\end{equation*}
 
    Dermed kommer $79\%$ igennem prøven\\
Hvis $1\%$ er bly kan vi beskrive det som $\frac{I(z_1+z_2)}{I_0}=e^{-\mu_1\cdot z_1-\mu_2 z_2}$ hvor $z_1=0,99cm$ og $z_2=0,01cm$. Vi finder $\mu$ for bly, vi finder i nist at $3,032\cdot10^1$ og densiteten af bly findes til $10,66\frac{g}{cm^3}$
\begin{equation*}
    \mu=2,032\cdot10^{1}\frac{g}{cm^2}\cdot10,66\frac{cm^3}{g}=21216,6cm^{-1}
\end{equation*}
$I(z)\over Z_0$ findes
\begin{equation*}
    \frac{I(z)}{I_0}=e^{-0,24cm^{-1}\cdot 0,99cm - 21216,6cm^{-1}\cdot 0,01cm}=0,09
\end{equation*}
Dermed kommer $9\%$ af strålingen igennem




\subsection{\bold{Opgave 6}}
\subsubsection*{Den direkte røntgenstråle fra en moderne synkrotronkilde kan have op til $10^{17} \frac{fotoner}{sekund}$ med en foton-energi på $30keV$. Hvad er effekten $P$, udtrykt i watt, af en sådan stråle?}
For at finde effekten skal vi finde strålens samlet energi pr. s. Derfor skal vi multiplicerer antallet af fotoner pr. sekund med foton-energien 
\begin{equation*}
P=10^{17}*\frac{fotoner}{sekund}\cdot30\cdot10^{3}eV\cdot1.602\cdot10^{-19} \frac{J}{eV}=480.6W
\end{equation*}
    
\subsection{\bold{Opgave 7}}


\section{}

\newpage

%\bibliographystyle{plain} - hvis I bruger BibTeX lsningen
%\bibliography{skabelon_bib}

\begin{thebibliography}{} % det simple alternativ til BibTeX
\bibitem{collins2000}
Phillip G. Collins and Phaedon Avouris, Scientific American, \textbf{283}, 62 (2000)
\end{thebibliography}

\newpage 

\appendix
\section{Appendix-overskrift}
Denne tekst står i det første appendix

\section{Endnu en appendix-overskrift}
Denne tekst står i næste appendix





\end{document}
